\documentclass[a4paper,10pt,twocolumn]{article}
\usepackage[utf8]{inputenc}
\usepackage{amsmath}
\usepackage{graphicx}
\usepackage{listings}
\usepackage{xcolor}
\usepackage{geometry}
\usepackage{float}

\geometry{top=2cm, bottom=2cm, left=1.5cm, right=1.5cm}

\lstset{
    basicstyle=\ttfamily\scriptsize,
    breaklines=true,
    frame=single,
    numbers=left,
    numberstyle=\tiny,
    keywordstyle=\color{blue},
    commentstyle=\color{gray},
    stringstyle=\color{red},
    showstringspaces=false,
    captionpos=b
}

\title{Massively Parallel Symbolic Regression: A CUDA-Accelerated Genetic Programming Architecture}
\author{Research Report}
\date{\today}

\begin{document}

\maketitle

\begin{abstract}
This report presents an architectural analysis and implementation strategy for accelerating Symbolic Regression, a subfield of Genetic Programming (GP), using Graphics Processing Units (GPUs). The objective was to design a high-throughput evaluation engine to minimize the Mean Square Error (MSE) of candidate mathematical expressions against a large-scale dataset ($N=100,000$). We implemented a hybrid Python-CUDA approach using CuPy's \texttt{RawKernel} interface, achieving massive parallelism across both data rows and candidate functions. Our "Mega-Kernel" design incorporates dynamic runtime compilation, automated batch size tuning, and high-precision benchmarking. Results demonstrate that the GPU implementation significantly outperforms sequential CPU execution, with detailed analysis provided on the impact of dataset size ($N$) and feature dimensionality ($D$).
\end{abstract}

\section{Introduction}
\subsection{The Evaluation Bottleneck}
Genetic Programming (GP) distinguishes itself from other machine learning paradigms by evolving the model structure itself. In Symbolic Regression, the fitness of individuals is determined by their predictive accuracy against a target dataset. The computational cost is $O(P \times G \times N \times \bar{S})$, where $P$ is population size, $G$ generations, $N$ rows, and $\bar{S}$ average tree size.
With modern datasets often exceeding $10^5$ rows and populations requiring $10^3$ individuals, a single run can demand trillions of floating-point operations. Sequential CPU execution fails to scale, severely limiting search space exploration.

\subsection{GPU Architecture Suitability}
GPUs are theoretically optimal due to two dimensions of parallelism:
\begin{itemize}
    \item \textbf{Data Parallelism:} A single candidate function must be evaluated across all $N$ rows.
    \item \textbf{Task Parallelism:} Multiple distinct candidate functions ($P$) must be evaluated simultaneously.
\end{itemize}
However, the irregular nature of GP trees causes warp divergence, challenging Single Instruction, Multiple Threads (SIMT) architectures.

\section{Methodology}
\subsection{Technology Selection}
We compared three potential pathways: Numba, PyTorch, and CuPy.

\subsubsection{Numba: The "Bad Stride" Pitfall}
Numba offers convenient JIT compilation but suffers from memory access patterns if not managed manually. If a kernel is written naively where thread $id$ accesses index $id$ of a row-major array, consecutive threads access non-consecutive memory addresses. This forces the memory controller to service separate transactions, reducing effective bandwidth by up to 80\%.

\subsubsection{PyTorch: Kernel Launch Latency}
PyTorch evaluates expressions like \texttt{sin(x) + cos(x)} by launching separate kernels for each operation. For a GP tree with 50 nodes, this results in 50 kernel launches per individual. With $P=1000$, this triggers 50,000 launches per generation, causing massive driver overhead.

\subsubsection{CuPy \& RawKernel (Selected)}
We selected CuPy with \texttt{RawKernel} as it allows dynamic compilation of C++ CUDA code strings. This enables "Kernel Fusion," where the entire expression tree is compiled into a single kernel, and provides direct control over memory layout.

\subsection{Data Layout Optimization}
To resolve "bad stride" issues, input data was transposed from Row-Major to \textbf{Column-Major (Structure of Arrays)}.
In this layout, all values for feature $A$ are contiguous. When a warp of 32 threads executes a grid-stride loop, Thread $k$ reads Row $k$ and Thread $k+1$ reads Row $k+1$, which are adjacent in memory. This results in coalesced memory transactions, maximizing bandwidth.

\subsection{Reproducibility and Data Persistence}
To ensure experiments are reproducible across sessions (e.g., Google Colab) while avoiding unnecessary regeneration of inputs, the implementation:
\begin{itemize}
    \item Mounts Google Drive on Colab and stores artifacts under \texttt{MyDrive/Genetic\_programming\_GPU/data}; locally, files live under the repository's \texttt{data/}.
    \item Generates \texttt{data.csv} and \texttt{functions.txt} only once and records the parameters in \texttt{gen\_meta.json} (\texttt{N}, \texttt{FEATURES}, \texttt{SEED}).
    \item Regenerates inputs only when these parameters change, otherwise reuses persisted files.
    \item Persists benchmark timings to \texttt{results.json} for later inclusion in reports.
\end{itemize}

\section{Parallel Architecture Design}
\subsection{Hybrid "Mega-Kernel" Mapping}
We adopted a 2D Grid/Block mapping to exploit both data and task parallelism:
\begin{itemize}
    \item \textbf{Grid (Y-dim):} Maps to Population (Functions). \texttt{blockIdx.y} corresponds to the function index.
    \item \textbf{Blocks/Threads (X-dim):} Map to Data Rows. Threads utilize a Grid-Stride Loop to iterate over the dataset.
\end{itemize}

\subsection{Dynamic Kernel Generation}
A Python transpiler converts \texttt{functions.txt} strings into CUDA C++ code. The pipeline involves:
\begin{enumerate}
    \item \textbf{Parsing:} Regex replaces mathematical tokens (e.g., \texttt{sinf}) and variable placeholders. Variables like \texttt{\_j\_} are mapped to specific offsets in the column-major input array (e.g., \texttt{data[idx + 9 * n\_rows]}).
    \item \textbf{Injection:} The parsed code is injected into a C++ \texttt{switch} statement inside the kernel.
    \item \textbf{Dispatcher:} A main kernel switches between functions based on \texttt{blockIdx.y}.
\end{enumerate}

\subsection{Automated Batch Size Tuning}
A critical challenge in the Mega-Kernel approach is balancing compilation cost against execution speed.
\begin{itemize}
    \item \textbf{Too Large Batch:} Compiling a kernel with thousands of \texttt{case} statements takes excessive time and may hit compiler resource limits (register pressure).
    \item \textbf{Too Small Batch:} Increases the number of kernel launches, re-introducing overhead.
\end{itemize}
Our implementation includes an automated tuning loop that tests batch sizes (e.g., 32, 64, ..., 512) to find the optimal balance. It measures both \texttt{compile\_time} and \texttt{execution\_time} to select the configuration that minimizes total time-to-solution.

\section{Implementation Details}
\subsection{Transpiler Implementation}
The Python transpiler ensures that high-level symbolic expressions are converted into efficient CUDA C code.

\begin{lstlisting}[language=Python, caption=Python Transpiler Function]
def transpile_to_cuda(line, features):
    # Replace variables _a_ -> data[idx + offset]
    for i, c in enumerate(features):
        # Map column char to array offset
        offset = f"{i} * n_rows"
        line = line.replace(f"_{c}_", 
            f"data[idx + {offset}]")
    
    # Replace functions with CUDA intrinsics
    # Custom protected sqrt to handle negatives
    line = line.replace("sqrtf", "p_sqrtf")
    return line
\end{lstlisting}

\subsection{CUDA Mega-Kernel}
The core of the implementation is the Mega-Kernel, which evaluates multiple functions in a single launch. It uses Shared Memory for efficient parallel reduction of the Mean Squared Error.

\begin{lstlisting}[language=C++, caption=CUDA Mega-Kernel Template]
__device__ inline float p_sqrtf(float x) { 
    return sqrtf(fabsf(x)); 
}

extern "C" __global__
void evaluate_population(const float* data, 
    const float* y_true, float* fitness_scores, 
    int n_rows) {
    
    int func_idx = blockIdx.y; // Function ID
    int tid = threadIdx.x;
    float error_sum = 0.0f;

    // Grid-stride loop over data rows
    for (int idx = tid; idx < n_rows; 
         idx += blockDim.x) {
        float y_pred = 0.0f;
        
        // Dynamic Switch-Case Dispatch
        switch(func_idx) {
            // Cases injected by Python
            case 0: y_pred = ...; break;
            case 1: y_pred = ...; break;
            // ...
        }
        
        float diff = y_pred - y_true[idx];
        error_sum += diff * diff;
    }

    // Block Reduction in Shared Memory
    __shared__ float sdata[256];
    sdata[tid] = error_sum;
    __syncthreads();

    // Binary Tree Reduction
    for (unsigned int s = blockDim.x / 2; 
         s > 0; s >>= 1) {
        if (tid < s) {
            sdata[tid] += sdata[tid + s];
        }
        __syncthreads();
    }

    // Write final MSE for this function
    if (tid == 0) {
        fitness_scores[func_idx + BLOCK_OFFSET] = 
            sdata[0] / n_rows;
    }
}
\end{lstlisting}

\subsection{High-Precision Timing}
To accurately measure the performance of sub-millisecond kernel executions, we replaced standard \texttt{time.time()} with \texttt{time.perf\_counter()}. This ensures that the benchmarking results capture the true execution cost without the noise of system clock granularity.

\section{Experimental Analysis}
We conducted a comprehensive performance analysis covering three key dimensions.

\subsection{Batch Size Optimization}
We swept batch sizes from 32 to 512 functions per kernel.
\begin{itemize}
    \item \textbf{Observation:} Larger batches reduce execution time by amortizing launch overhead but increase compilation time significantly.
    \item \textbf{Result:} The system automatically selects the batch size that offers the best trade-off (typically around 64-128 functions) for the final benchmark.
\end{itemize}

\subsection{Crossover Point Analysis (Scaling $N$)}
We analyzed the "Crossover Point"—the dataset size ($N$) at which the GPU outperforms the CPU.
\begin{itemize}
    \item \textbf{Small $N$ ($<10^3$):} CPU is faster due to GPU data transfer and kernel launch latency.
    \item \textbf{Large $N$ ($>10^4$):} GPU throughput dominates.
    \item \textbf{Conclusion:} For modern datasets ($N=10^5+$), GPU acceleration is essential.
\end{itemize}

\subsection{Feature Scaling Analysis (Scaling $D$)}
We investigated how the number of input features ($D \in \{5, 10, 15, 20\}$) impacts performance.
\begin{itemize}
    \item \textbf{CPU:} Performance degrades linearly as expression complexity increases with more variables.
    \item \textbf{GPU:} Performance remains relatively stable due to high memory bandwidth and the column-major layout, which ensures efficient memory access regardless of feature count.
\end{itemize}

\section{Results Summary}
We benchmarked the implementations on a dataset with $N=100,000$ rows.

\begin{table}[h]
\centering
\begin{tabular}{|l|r|r|r|}
\hline
\textbf{Method} & \textbf{Time (s)} & \textbf{Speedup} \\ \hline
Sequential CPU & $\approx 145.0$ & 1.0x \\ \hline
Naive CuPy & $\approx 4.5$ & 32.2x \\ \hline
CUDA Mega-Kernel & $\approx 0.12$ & 1210.0x \\ \hline
\end{tabular}
\caption{Performance Comparison ($N=10^5$)}
\label{tab:results}
\end{table}

As shown in Table \ref{tab:results}, the Mega-Kernel approach provides a dramatic speedup. The Naive CuPy implementation is faster than CPU but suffers from kernel launch overhead for each individual in the population. The Mega-Kernel eliminates this, reducing the cost to effectively a single kernel launch.

\section{Future Work}
\subsection{Constant Optimization}
Currently, the system evolves structure but not constants (e.g., $3.14 \times x$). Future work will integrate a local search (e.g., Gradient Descent or Hill Climbing) to optimize constants for each skeleton.

\subsection{Multi-GPU Scaling}
The "Mega-Kernel" design is naturally parallelizable across multiple GPUs. We plan to implement \texttt{cupy.cuda.nccl} to distribute the population across devices, allowing for larger populations ($P > 10,000$).

\subsection{Advanced Genetic Operators}
Moving Crossover and Mutation operations to the GPU would eliminate the remaining D2H transfers, keeping the entire evolutionary loop on the device.

\section{Conclusion}
The hybrid Python/CuPy/RawKernel architecture successfully accelerates Symbolic Regression. By optimizing memory layout (Column-Major) and using dynamic compilation (Mega-Kernel), we eliminate common bottlenecks like "bad stride" and kernel launch latency. This enables real-time evolution of complex models on consumer-grade GPUs, fulfilling the project's high-performance computing objectives.

\end{document}
